\documentclass[a4paper,11pt]{article}

\usepackage{wrapfig}
\usepackage{amsmath}
\usepackage{listings}
\usepackage{multicol}
\usepackage[dvipsnames]{xcolor}
\usepackage{graphicx}
\usepackage[left=2.5cm, right=1.5cm, vmargin=2.5cm]{geometry}
%Russian-specific packages
%--------------------------------------
\usepackage[T2A]{fontenc}
\usepackage[utf8]{inputenc}
\usepackage[russian, english]{babel}
%--------------------------------------
\graphicspath{ {./img/} }

\lstdefinestyle{mycode} {
    backgroundcolor=\color{White},   
    commentstyle=\color{ForestGreen},
    keywordstyle=\color{Blue},
    numberstyle=\tiny\color{Maroon},
    stringstyle=\color{Plum},
    basicstyle=\ttfamily\footnotesize,
    breakatwhitespace=false,         
    breaklines=true,                 
    captionpos=b,                    
    keepspaces=true,                 
    numbers=left,                    
    numbersep=5pt,                  
    showspaces=false,                
    showstringspaces=false,
    showtabs=false,                  
    frame=single,
    tabsize=2,
}

\begin{document}
\lstset{style=mycode}

\begin{titlepage}
  \center
  ФЕДЕРАЛЬНОЕ ГОСУДАРСТВЕННОЕ АВТОНОМНОЕ ОБРАЗОВАТЕЛЬНОЕ УЧЕРЕЖДЕНИЕ ВЫСШЕГО ОБРАЗОВАНИЯ\linebreak  
  «Санкт-Петербургский политехнический университет Петра Великого»\\[2cm] 
  \textsc{\Large Институт компьютерных наук и технологий}\\[6.5cm]
  
  { 
    \huge \bfseries Курсовой проект\\
    по дисциплине «Функциональное программирование»
  } \\[6.5cm]

  \begin{multicols}{2}
  \begin{flushright}
    \large
    
    {Выполнил студент гр. 3530904/80001:}\\[0.5cm]
    
    {Руководитель\\
    ассистент ВШПИ}

  \end{flushright}
  \begin{flushright}
    
    {Бареков А. М.}\\[0.5cm]   
     
    Лукашин А. А.
    
  \end{flushright}
  \end{multicols}
  
  \bigskip
  \centering
  {
    Санкт-Петербург\\
    2019
  }
  \vfill
\end{titlepage}

\section{Задание}
  Калькулятор, поддерживающий простые арифметические операции, приоритеты и скобки.
  %я хотел что-то поинтереснее, но времени совсем нет, кстати :'(((

\section{Ход работы}
  \subsection{Алгоритм решения}
  Выражение, заданное в инфиксной нотации с помощью алгоритма сортировочной станции (\textbf{Shunting Yard Algorithm}) переводится в
  обратную польскую нотацию, а затем вычисляется с помощью левоассоциативной свёртки получившегося списка токенов.
  Код программы приведён в приложении.
  \subsection{Скриншот}
    \includegraphics[scale=1.0]{img.png}

\section{Выводы}
  В ходе работы был изучен функциональный подход к программированию, который значительно отличается от стандартного императивного подхода.
  Изучены некоторые основные алгоритмы, используемые в функциональном программировании и произведена работа с ними.
  \newpage

\section{Приложение}
  \subsection{Код файла main.hs}
    \lstinputlisting[language=Haskell]{../main.hs}
  \subsection{Код модуля calc.hs}
    \lstinputlisting[language=Haskell]{../calc.hs}

\end{document}
